\documentclass{article}

\title{Markov Chain Monte Carlo }
\author{Ben Larson, u0625822}
\usepackage{Sweave}
\begin{document}
\input{Homework3-concordance}


\section{Image denoising }

\begin{Schunk}
\begin{Sinput}
> require(png)
> ## Reads a PNG image (as a simple matrix), which for some reason needs to be
> ## transposed and flipped to be displayed correctly
> read.image = function(filename)
+ {
+   # y = readPNG(system.file(filename,package="png"), TRUE)
+   y = readPNG(filename)
+   print(min(y))
+   print(max(y)) 
+   ## This is just how I encoded pixel intensities
+   ## After this, the labels will be black = -1 and white = +1
+   y = y * 20 - 10
+   # y = y/max(y) 
+   n = nrow(y)
+   
+   ## transpose and flip image
+   t(y)[,n:1]
+ }
> ## Displays an image
> display.image = function(x, col=gray(seq(0,1,1/256)))
+ {
+   w = dim(x)[1]
+   h = dim(x)[2]
+   par(mai = c(0,0,0,0))
+   image(x, asp=h/w, col=col)
+ }
> ##Function from http://stackoverflow.com/questions/36073795/get-neighbors-function-in-r
> ## returns up down, right, left 
> get.nbhd <- function(m, i, j) {
+   # get indices
+   idx <- matrix(c(i-1, i+1, i, i, j, j, j+1, j-1), ncol = 2)
+   # set out of bound indices to 0
+   idx[idx[, 1] > nrow(m), 1] <- 0
+   idx[idx[, 2] > ncol(m), 2] <- 0
+   # return (m[idx])
+   return (idx)
+ }
> ising = function(x,alpha,beta) 
+ {
+   # browser()
+   col = dim(x)[1]
+   row = dim(x)[2]
+   x_ = matrix(rbinom(row*col,1,0.5),nrow = col, ncol = row) 
+   x_[x_==0] = -1 
+   for (i in 1:nrow(x)){
+     for (j in 1:ncol(x)){
+       n = get.nbhd(x,i,j)
+       b = beta *sum(x[n])*x[i,j]
+       a = alpha *x[i,j]
+       x_[i,j] = -a - b 
+     }
+   }
+   return (x_) 
+  # apply(M, 1, fun)
+  # apply(M, 2, fun)
+ }
> ising_gibb = function(x,alpha,beta,sigma) 
+ {
+   col = dim(x)[1]
+   row = dim(x)[2]
+   x_ = matrix(rbinom(row*col,1,0.5),nrow = col, ncol = row) 
+   x_[x_==0] = -1 
+   y = x
+   # y = matrix(0, nrow=col, ncol=row) 
+   vL = c(0) 
+   listIm = list() 
+   for (itr in 1:20) {
+     # print(x == x_)
+     x = x_
+     # display.image(as.matrix(x))
+     for (i in 1:nrow(x)){
+       for (j in 1:ncol(x)){
+         n = get.nbhd(x,i,j)
+         b = beta*sum(x[n])*x[i,j] 
+         a = alpha*x[i,j] 
+         l = 1/(2*sigma^2)*(x[i,j]-y[i,j])^2
+        
+         # P(xi|x = 1)
+         a1 = alpha*1
+         b1 = beta*sum(x[n])
+         l1 = 1/(2*sigma^2)*(1-y[i,j])^2
+         # P(xi|x = -1) 
+         an = alpha*(-1)
+         bn = beta*sum(x[n])*(-1)
+         ln = 1/(2*sigma^2)*(-1-y[i,j])^2
+         
+         n1 = exp(-1*(-a1 - b1 + l1)) # x = +1
+         n2 = exp(-1*(-an - bn + ln)) # x = -1
+         xi = exp(-1*(-a- b+ l))
+         
+         p1 = xi/(n1+n2)
+         p_1 = 1-p1
+         
+         if(x[i,j] == 1)
+         {
+           x_[i,j] = sample(c(1,-1), 1, prob = c(p1,p_1))
+         }else{
+           x_[i,j] = sample(c(-1,1), 1, prob = c(p1,p_1))
+         }
+       }
+     }
+     listIm[[itr]] = x_
+     sigma = sqrt(var(as.vector(x_-y)))
+     v = 1/length(x_)*sum((x_-y)^2) 
+     print(v) 
+     vL = c(vL, v)
+   }
+   # listIm = tail(listIm,-1) 
+   vL = tail(vL,-1) 
+   plot(vL) 
+   listIm = tail(listIm, 6-length(listIm))
+   print(sigma) 
+   return (listIm)
+ }
> #Create a binomial matrix matrix(rbinom(100,1,0.5),nrow = 100, ncol = 100) 
> # require(knitr)
> # # dev.off() 
> # opts_knit$set(root.dir= '.')
> 
> # (WD <- getwd())
> # if(!is.null(WD))setwd(WD) 
> this.dir <- dirname(parent.frame(2)$ofile)